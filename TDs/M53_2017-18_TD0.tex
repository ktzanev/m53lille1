\documentclass[a4paper,12pt,reqno]{amsart}
\usepackage{macros_M53}

\begin{document}

\hautdepage{TD0: Géométrie plane et nombres complexes}

%-----------------------------------
\begin{exo} (Droites)

  \begin{enumerate}
    \item\label{q:reel} On considère l'espace affine $\mathbb{R}^{2}$. Soit $\mathcal{D}$ la droite $\big\{(x,y) \;\big|\; x+2y=1\big\}$.
    \begin{enumerate}
      \item Déterminer toutes les équations cartésiennes qui définissent la même droite $\mathcal{D}$.
      \item Déterminer toutes les équations cartésiennes qui définissent une droite parallèle à $\mathcal{D}$.
      \item Déterminer la droite vectorielle $\vv{\mathcal{D}}$ parallèle à $\mathcal{D}$.
      \item Déterminer les équations cartésiennes qui définissent la droite parallèle à $\mathcal{D}$ qui passe par $A=(2,1)$.
      \item Déterminer une équation de la droite qui passe par les points $C=(2,3)$ et $D=(4,-3)$. Est-elle parallèle à $\mathcal{D}$?
    \end{enumerate}
    \item On considère le plan complexe $\mathbb{C}$.
    \begin{enumerate}
      \item Montrer que toute droite réelle de $\mathbb{C}$ est définie par une équation complexe de la forme
        \[
          \big\{z \in \mathbb{C} \;\big|\; \overline{\beta}z+\beta\overline{z}+\gamma=0\big\}
        \]
        où $\beta \in \mathbb{C}^{*}$ et $\gamma \in \mathbb{R}$.
      \item Déterminer une équation complexe de la droite $\mathcal{D}$ de la question (\ref{q:reel}).
      \item Donner une condition sur les coefficients des équations complexes pour que deux droites soient parallèles.
    \end{enumerate}
  \end{enumerate}
\end{exo}

%-----------------------------------
\begin{exo} (Conditions géométriques)
  \begin{enumerate}
    \item Étant donnés deux points distincts $A$ et $B$ du plan complexe d'affixes $a$ et $b$, donner une condition sur $a\overline{b}$ pour que la droite $AB$ passe par $O$ d'affixe $0$. Préciser quand $O$ est entre $A$ et $B$, et quand il ne l'est pas.

    \item Donner une condition nécessaire et suffisante sur les affixes $a$, $b$, $c$ de trois points $A$, $B$, $C$ du plan complexe pour que le triangle $ABC$ soit équilatéral.
  \end{enumerate}
\end{exo}

%-----------------------------------
\begin{exo} (Équation d'un cercle)

  Montrer que tout cercle du plan complexe est défini par une équation de la forme
    $$
      z\overline{z}-a\overline{z}-\overline{a}z+c=0,
    $$
  où $a$ est un nombre complexe et $c$ un réel vérifiant $c \leq |a|^{2}$. Montrer que réciproquement toute équation de ce type est celle d'un cercle.
\end{exo}

%-----------------------------------
\begin{exo} (Encore des équations)
  \begin{enumerate}
    \item Discuter selon les valeurs de $\alpha,\gamma \in \mathbb{R}$ et $\beta \in \mathbb{C}$ quel est le sous-ensemble de $\mathbb{C}$ défini par l'équation
      $$
        \alpha.z\overline{z}+\beta\overline{z}+\overline{\beta}z+\gamma=0.
      $$
    \item Soit $\lambda$ un nombre réel positif, décrire géométriquement l'ensemble
      $$
        E_{\lambda}=\lbrace z\in\mathbb{C}:\vert z-a \vert =\lambda \vert z-b \vert \rbrace .
      $$
    \item Déterminer l'ensemble des nombres complexes tels que $\vert z-1 \vert=\vert z-i z\vert=\vert z-i \vert$.
  \end{enumerate}
\end{exo}

%-----------------------------------
\begin{exo}

  \image{l}{7cm}{-17mm}{0mm}{M53_2017-18_TD0_exo05.tikz}

  À l'extérieur d'un triangle $ABC$, on construit trois carrés de bases les côtés et de centres $P$, $Q$ et $R$. Montrer que les segments $AP$ et $QR$ (resp. $BQ$ et $RP$, $CR$ et $PQ$) sont orthogonaux et de même longueur. En déduire que les droites $AP$, $BQ$ et $CR$ sont concourantes.
\end{exo}

%-----------------------------------
\begin{exo}

  \image{r}{7cm}{-17mm}{0mm}{M53_2017-18_TD0_exo06.tikz}

  On construit à l'extérieur d'un parallélogramme $ABCD$ quatre carrés de bases les côtés et de centres $M$, $N$, $P$ et $Q$. Montrer que $MNPQ$ est un carré.
\end{exo}


\end{document}
